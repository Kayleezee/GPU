%----------------------------------------------------------------------------------------
%	PACKAGES AND OTHER DOCUMENT CONFIGURATIONS
%----------------------------------------------------------------------------------------

\documentclass{article}

\usepackage{fancyhdr} % Required for custom headers
\usepackage{lastpage} % Required to determine the last page for the footer
\usepackage{extramarks} % Required for headers and footers
\usepackage[usenames,dvipsnames]{color} % Required for custom colors
\usepackage{graphicx} % Required to insert images
\usepackage{listings} % Required for insertion of code
\usepackage{courier} % Required for the courier font
\usepackage{lipsum} % Used for inserting dummy 'Lorem ipsum' text into the template
\usepackage[utf8]{inputenc}
\usepackage[ngerman]{babel}

% Margins
\topmargin=-0.45in
\evensidemargin=0in
\oddsidemargin=0in
\textwidth=6.5in
\textheight=9.0in
\headsep=0.25in

\linespread{1.1} % Line spacing

% Set up the header and footer
\pagestyle{fancy}
%\lhead{\hmwkAuthorName} % Top left header
\chead{\hmwkClass\ : \hmwkTitle} % Top center head
\rhead{\firstxmark} % Top right header
\lfoot{\lastxmark} % Bottom left footer
\cfoot{} % Bottom center footer
\rfoot{Page\ \thepage\ of\ \protect\pageref{LastPage}} % Bottom right footer
\renewcommand\headrulewidth{0.4pt} % Size of the header rule
\renewcommand\footrulewidth{0.4pt} % Size of the footer rule

\setlength\parindent{0pt} % Removes all indentation from paragraphs

%----------------------------------------------------------------------------------------
%	CODE INCLUSION CONFIGURATION
%----------------------------------------------------------------------------------------

\definecolor{MyDarkGreen}{rgb}{0.0,0.4,0.0} % This is the color used for comments
\lstloadlanguages{Perl} % Load Perl syntax for listings, for a list of other languages supported see: ftp://ftp.tex.ac.uk/tex-archive/macros/latex/contrib/listings/listings.pdf
\lstset{language=Perl, % Use Perl in this example
        frame=single, % Single frame around code
        basicstyle=\small\ttfamily, % Use small true type font
        keywordstyle=[1]\color{Blue}\bf, % Perl functions bold and blue
        keywordstyle=[2]\color{Purple}, % Perl function arguments purple
        keywordstyle=[3]\color{Blue}\underbar, % Custom functions underlined and blue
        identifierstyle=, % Nothing special about identifiers                                         
        commentstyle=\usefont{T1}{pcr}{m}{sl}\color{MyDarkGreen}\small, % Comments small dark green courier font
        stringstyle=\color{Purple}, % Strings are purple
        showstringspaces=false, % Don't put marks in string spaces
        tabsize=5, % 5 spaces per tab
        %
        % Put standard Perl functions not included in the default language here
        morekeywords={rand},
        %
        % Put Perl function parameters here
        morekeywords=[2]{on, off, interp},
        %
        % Put user defined functions here
        morekeywords=[3]{test},
       	%
        morecomment=[l][\color{Blue}]{...}, % Line continuation (...) like blue comment
        numbers=left, % Line numbers on left
        firstnumber=1, % Line numbers start with line 1
        numberstyle=\tiny\color{Blue}, % Line numbers are blue and small
        stepnumber=5 % Line numbers go in steps of 5
}

% Creates a new command to include a perl script, the first parameter is the filename of the script (without .pl), the second parameter is the caption
\newcommand{\perlscript}[2]{
\begin{itemize}
\item[]\lstinputlisting[caption=#2,label=#1]{#1.pl}
\end{itemize}
}

%----------------------------------------------------------------------------------------
%	DOCUMENT STRUCTURE COMMANDS
%	Skip this unless you know what you're doing
%----------------------------------------------------------------------------------------

% Header and footer for when a page split occurs within a problem environment
\newcommand{\enterProblemHeader}[1]{
%\nobreak\extramarks{#1}{#1 continued on next page\ldots}\nobreak
%\nobreak\extramarks{#1 (continued)}{#1 continued on next page\ldots}\nobreak
}

% Header and footer for when a page split occurs between problem environments
\newcommand{\exitProblemHeader}[1]{
%\nobreak\extramarks{#1 (continued)}{#1 continued on next page\ldots}\nobreak
%\nobreak\extramarks{#1}{}\nobreak
}

\setcounter{secnumdepth}{0} % Removes default section numbers
\newcounter{homeworkProblemCounter} % Creates a counter to keep track of the number of problems

\newcommand{\homeworkProblemName}{}
\newenvironment{homeworkProblem}[1][Problem \arabic{homeworkProblemCounter}]{ % Makes a new environment called homeworkProblem which takes 1 argument (custom name) but the default is "Problem #"
\stepcounter{homeworkProblemCounter} % Increase counter for number of problems
\renewcommand{\homeworkProblemName}{#1} % Assign \homeworkProblemName the name of the problem
\section{\homeworkProblemName} % Make a section in the document with the custom problem count
%\enterProblemHeader{\homeworkProblemName} % Header and footer within the environment
}{
%\exitProblemHeader{\homeworkProblemName} % Header and footer after the environment
}

\newcommand{\problemAnswer}[1]{ % Defines the problem answer command with the content as the only argument
\noindent\framebox[\columnwidth][c]{\begin{minipage}{0.98\columnwidth}#1\end{minipage}} % Makes the box around the problem answer and puts the content inside
}

\newcommand{\homeworkSectionName}{}
\newenvironment{homeworkSection}[1]{ % New environment for sections within homework problems, takes 1 argument - the name of the section
\renewcommand{\homeworkSectionName}{#1} % Assign \homeworkSectionName to the name of the section from the environment argument
\subsection{\homeworkSectionName} % Make a subsection with the custom name of the subsection
%\enterProblemHeader{\homeworkProblemName\ [\homeworkSectionName]} % Header and footer within the environment
}{
%\enterProblemHeader{\homeworkProblemName} % Header and footer after the environment
}

%----------------------------------------------------------------------------------------
%	NAME AND CLASS SECTION
%----------------------------------------------------------------------------------------

\newcommand{\hmwkTitle}{Übung\ \#4} % Assignment title
\newcommand{\hmwkDueDate}{Donnerstag,\ 27.\ November\ 2014} % Due date
\newcommand{\hmwkClass}{GPU Computing} % Course/class
\newcommand{\hmwkClassTime}{} % Class/lecture time
\newcommand{\hmwkClassInstructor}{} % Teacher/lecturer
\newcommand{\hmwkAuthorName}{Günther Schindler, Alexander Schnapp, Klaus Naumann} % Your name

%----------------------------------------------------------------------------------------
%	TITLE PAGE
%----------------------------------------------------------------------------------------

\title{
\vspace{2in}
\textmd{\textbf{\hmwkClass:\ \hmwkTitle}}\\
\normalsize\vspace{0.1in}\small{Abgabe\ am\ \hmwkDueDate}\\
\vspace{0.1in}\large{\textit{\hmwkClassTime}}
\vspace{3in}
}

\author{\textbf{\hmwkAuthorName}}
\date{} % Insert date here if you want it to appear below your name

%----------------------------------------------------------------------------------------

\begin{document}

\maketitle

%----------------------------------------------------------------------------------------
%	TABLE OF CONTENTS
%----------------------------------------------------------------------------------------

%\setcounter{tocdepth}{1} % Uncomment this line if you don't want subsections listed in the ToC
\newpage
\tableofcontents
\newpage

%----------------------------------------------------------------------------------------
%	Reading
%----------------------------------------------------------------------------------------

\begin{homeworkProblem}[Reading]
\subsection{NVIDIA TESLA : A UNIFIED GRAPHICS AND COMPUTING ARCHITECTURE}
The authors Erik Lindholm, John Nickolls, Stuart Oberman and John Montrym discuss in 
'NVIDIA TESLA : A UNIFIED GRAPHICS AND COMPUTING ARCHITECTURE' the requirements that
drove the unified graphics and parallel computing processor architecture, describe the 
Tesla architecture, and how it is enabling widespread deployment of parallel computing 
and graphics applications. Main reasons for developing the new architecture was to
enable flexible, programmable graphics and high performance computing. The primary
goal of this contribution is to introduce the new architecture and their capabilities.

Key insight of the new architecture is the new processor architecture called
single-instruction, multiple-thread (SIMT), which is similar to single-instruction,
multiple-data (SIMD) design, but applies one instruction to multiple independent 
threads in parallel. Due to the new architecture, each streaming multiprocessor 
is able to execute 24 warps (32 parallel threads). With that a streaming multiprocessor
manages and executes up to 768 concurrent threads in hardware with zero scheduling
overhead.

This contribution is more a promotional guide then an objective article about the NVIDIA 
Tesla architecture. However, the introduced architecture and their capabilities is 
ground-breaking for the high performance computing. Although the SIMT model needs
capable programmers and fixed calculation problems to attain success in computing.

\subsection{Debunking the 100X GPU vs. CPU Myth}
In this paper the author wants to verify wether the use of a GPU on throuphut computing with large amount of data-level parallislm results in a substanable speedup of 10X up to 100X compared to the use of a CPU.
 To acheive that he compares the perfomance of the Nvidia GTX280 GPU and a Intel Core i7-960 CPU on different workloads.
  He recognizes a speed-up that is heavily depending of the  current workload, but in average the GPU shows a speed-up of only about 2.5X, so markebly lower than the prediction of other papers.

His first reason for this is that you have to make sure you compare your probably optimized GPU code with an also opimized CPU-version.
Thereby also the different possibilities of opimising on CPUs (multithreading \& cache blocking) and GPUs (minimizing global synchronization \& using local shared buffers) are discussed.
His second reason is that you also use compareble hardware components, so not compare e.g. a high performance
GPU to a mobile CPU. 
The violation of both could leed to the predicted optimistic speed-ups. 

Over all the argumentation of the author, why the predicted speed-up  of 10-1000X is very much lower, is quite reasonable, but as a reader it is hard to believe that all the other papers are neglegting his rules of 'compare equal to equal' in such a hefty way, since this should actually be clear as a rule.
\end{homeworkProblem}
\pagebreak
%----------------------------------------------------------------------------------------
%	Shared Memory Analysis - Basics
%----------------------------------------------------------------------------------------

\begin{homeworkProblem}[Shared Memory Analysis - Basics]
Für die Analyse des Shared Memory wurde Daten von Global Memory in das Shared Memory
und umgekehrt kopiert. Für die Testreihen wurden zunächst nur ein Thread-Block verwendet.
\begin{center}
\includegraphics[width=0.8\columnwidth]{glob2share.png}
\end{center}
Beim Kopieren von Global Memory zu Shared Memory hängt der Durchsatz stark von der
verwendeten Threadanzahl ab. Selbst ab einer Threadanzahl von 512 sieht man noch
eine deutliche Steigerung des Durchsatzes.
\begin{center}
\includegraphics[width=0.8\columnwidth]{share2glob.png}
\end{center}
Beim Lesen aus dem Shared Memory in das Global Memory wird der maximale Durchsatz
beinahe ab einer Thradanzahl von 512 erreicht. Für eine höhere Threadanzahl bleibt
der Durchsatz konstant.

Die Kombination von 768 Threads per Block und einer Blockanzahl von 14 erreicht bei 
einer Problemgröße von 49k Byte einen maximalen Durchsatz von etwa 24 GB/s.
\\\\
Für das Lesen aus dem Shared Memory in den Register und umgekehrt wurden ebenfalls
zunächst bei nur einem Threadblock die Threadanzahl variiert.

Beim Lesen aus dem Shared Memory wird ein Durchsatz von fast 16 GB/s erreicht, was
eine deutliche Steigerung im Vergleich zum Global Memory aufzeigt. Dieser Durchsatz
wird ab einer Threadanzahl von 512 erreicht. Danach bleibt der Durchsatz beinahe
konstant.
\begin{center}
\includegraphics[width=0.8\columnwidth]{share2reg.png}
\end{center}
Wird von dem Register in das Shared Memory kopiert sieht der Verlauf relativ
identisch aus. Der maximale Durchsatz beträgt hier 15 GB/s. 
\begin{center}
\includegraphics[width=0.8\columnwidth]{reg2share.png}
\end{center}
Beim variieren der Thread und Blockanzahl wird ebenfalls (wie bei global to shared)
ein maximaler Durchsatz von 24 GB/s bei einer Threadanzahl von 512 und einer
Blockanzahl von 7 erreicht.
\\\\
Schlussfolgernd lässt sich sagen, dass der Lese- und Schreibezugriff zu Shared Memory
deutlich schneller ist als zu Lese- und Schreibezugriff (bei einem Block). Werden 
mehrere Blöcke eingesetzt erreicht wird ein identischer Durchsatz von 24 GB/s für beide
Memory Typen erreicht. Jedoch wird dieser bei Shared Memory bei geringer Threadanzahl
und Blockanzahl erreicht. Dies lässt vermuten, dass mit steigender Problemgröße der
Unterschied zwischen den Durchsätzen noch deutlicher zu sehen ist.
\end{homeworkProblem}
%----------------------------------------------------------------------------------------
%	Shared Memory Analysis - Conflicts
%----------------------------------------------------------------------------------------
\begin{homeworkProblem}[Shared Memory Analysis - Conflicts]
Folgender Kernel lädt Daten von Shared Memory in das Thread-lokale Register. Jeder
Thread operiert genau auf einem Element vom Typ float. Die Zeit, bzw. die Taktzyklen, 
werden mit der clock64() funktion ermittelt. Für die Messungen werden 32 Threads per 
Block verwendet. Dies entspricht der Anzahl an Threads die in einer Gruppe 
zusammengefasst, gleichzeitig ausgeführt werden und dient somit am bestern der Analyse
der Bank Conflicts. Da multiple Thread-Blöcke auch ihre eigenen Shared Memory Bereiche
hervorrufen wird hier auch nur mit einem Thread-Block analysiert.
\begin{lstlisting}{c}
__global__ void 
bankConflictsRead(float * outFloat, int iStride, unsigned long long *ullTime)
{
  /* Static size of shared memory */
  __shared__ float s_memoryA[2024];
  /* Variable in register */
  float r_var;
  /* Start measure clock cycles */
  unsigned long long startTime = clock64();
  /* Access data from shared memory to register */
  r_var = s_memoryA[threadIdx.x*iStride];
  /* End measure clock cycles */
  *ullTime = clock64() - startTime;
  /* Conditionally assign register var, so it wont get optimized */
  if(threadIdx.x == 0) outFloat[0] = r_var;
}
int main()
{
  bankConflictsRead <<< 1, 32 >>> (outFloat, optStride, d_ullTime);
}
\end{lstlisting}

Shared Memory ist in verschiedene Bänke unterteilt. Jede dieser Bänke kann nur einen
Zugriff auf einmal verarbeiten. Wird innerhalb einer Warp Ausführung mehrmals auf eine 
Bank zugegriffen, wird der Zugriff serialisiert und benötigt somit mehr Zeit.

Folgender Plot zeigt die benötigte Clock Anzahl in Abhänigkeit von dem verwendeten
Stride. 
\begin{center}
\includegraphics[width=0.8\columnwidth]{conflicts.png}
\end{center}

Der Plot verdeutlicht, dass der Zugriff nur dann konfliktfrei abläuft, wenn der Stride
keinen gemeinsamen Teiler mit der Bankanzahl besitzt. So müssen bei einem Stride von 
2, 6, 10, 14, etc. zwei Zugriffe serialisert werden. Bei einem Stride 4, 12, 20, etc.
müssen vier Zugriffe serialisiert werden. Dieser Trend setzt sich bis zu einem Stride
von 32 fort, wobei alle 32 Zugriffe serialisert werden müssen.

Übersteigt der Stride die Anzahl von 32 fängt der Trend wieder von vorne an. Dies 
bedeutet, dass ein ganzes vielfaches der Bankanzahl erreicht ist und somit Shared
Memory in 32 Bänke unterteilt ist.
\end{homeworkProblem}

%----------------------------------------------------------------------------------------
%	Matrix Multiply – CPU sequential version
%----------------------------------------------------------------------------------------

\begin{homeworkProblem}[Matrix Multiply – CPU sequential version]
\subsection{Verify results}
In dem letzten Teil von Übung 4 wurde eine sequentielle Variante der Matrix-Multiplikation implementiert. Die Matrizen enthalten Werte des Datentyps \textit{double}.
Die Ergebnismatrix C enthält - für einen Input mit 5x5 Elementen - nach dem Ausführen der Matrix-Multiplikation folgende Werte:
 \begin{center}
\begin{tabular}{ |c|c|c|c|c|c| } 
\hline
\hline
C[i,j] & C[,0] & C[,1] & C[,2] & C[,3] & C[,4] \\
\hline
C[0,] & 0 & 30 & 60 & 90 & 120 \\ 
\hline
C[1,] & 0 & 40 & 80 & 120 & 160 \\ 
\hline
C[2,] & 0 & 50 & 100 & 150 & 200 \\ 
\hline
C[3,] & 0 & 60 & 120 & 180 & 240 \\
\hline
C[4,] & 0 & 70 & 140 & 210 & 280 \\
\hline
\hline
\end{tabular}
\end{center}
\subsection{Run time}
In der nächsten Tabelle sind die Ausführungszeiten in Abhängigkeit der Matrixdimensionen dargestellt. Die Messungen wurden hierbei mit \textit{getTimeOfDay()} durchgeführt:
\begin{lstlisting}{c}
/* Start time-measurement */
double dstartMeasurement(void)
{
  struct timeval tim;
  gettimeofday(&tim, NULL);
  return tim.tv_sec+(tim.tv_usec/1000000.0);
}

/* Stop time-measurement */
double dstopMeasurement(double dStartTime)
{
  struct timeval tim;
  gettimeofday(&tim, NULL);
  return (tim.tv_sec+(tim.tv_usec/1000000.0)) - dStartTime;
}
\end{lstlisting}
\begin{center}
\begin{tabular}{ |c|c|c|c|c|c| } 
\hline
\hline
Size & Time[s] \\
\hline
16 & 0.000025 \\ 
\hline
32 & 0.000192 \\ 
\hline
64 & 0.001518 \\ 
\hline
128 & 0.012200 \\
\hline
256 & 0.103475 \\
\hline
512 & 0.948771 \\
\hline
1024 & 9.688985 \\
\hline
2048 & 122.3314 \\
\hline
4096 & 1115.690 \\
\hline
\hline
\end{tabular}
\end{center}
\begin{center}
\includegraphics[width=0.7\columnwidth]{matrix_multiply.png}
\end{center}
Betrachtet man den Graphen der Ausführungszeiten und die dazugehörige Tabelle, stellt man fest, dass die Ausführungszeit in Abhängigkeit zu der Matrixgröße ansteigt. Bei jeder Verdopplung der Matrixgröße steigt die Ausführungszeit um einen Faktor von ca. 10. Ab einer gewissen Größe ist die Matrixgröße zu groß für den Cache und müssen somit in den Arbeitsspeicher geladen werden. Im Vergleich zum Cache ist die Übertragungszeit und der Arbeitsspeicher selbst erheblich langsamer und somit steigt die Zeit der Berechnung rapide an.
\subsection{GFLOP/s}
In der Aufgabe wurden quadratische Matrizen verwendet, wobei bei jeder Schleife 2 floating point Operationen druchgeführt werden. Daraus ergibt sich folgender Grad: 
\\${\mathcal O (2*n^3)}$. \\
Als Beispiel betrachten wir eine Matrixgröße von 1024 x 1024:
\\${\mathcal O (2*1024^3)}$. \\
Dies ergibt ca. \textit{2.147 GFLOP} welche durch die Ausführungszeit (siehe Tabelle) geteilt werden und somit \textit{2.217 MFLOP/s}.
Der theoretische \textit{Peek} der Prozessoren der creek-Server liegt jedoch weit höher. Diese Differenz ergibt sich durch die Matrixgröße und des bereits erwähnten langsameren Geschwindkeit des Arbeitsspeichers.
\end{homeworkProblem}
\pagebreak
\end{document}