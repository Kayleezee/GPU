%----------------------------------------------------------------------------------------
%	PACKAGES AND OTHER DOCUMENT CONFIGURATIONS
%----------------------------------------------------------------------------------------

\documentclass{article}

\usepackage{fancyhdr} % Required for custom headers
\usepackage{lastpage} % Required to determine the last page for the footer
\usepackage{extramarks} % Required for headers and footers
\usepackage[usenames,dvipsnames]{color} % Required for custom colors
\usepackage{graphicx} % Required to insert images
\usepackage{listings} % Required for insertion of code
\usepackage{courier} % Required for the courier font
\usepackage{lipsum} % Used for inserting dummy 'Lorem ipsum' text into the template
\usepackage[utf8]{inputenc}
\usepackage[ngerman]{babel}

% Margins
\topmargin=-0.45in
\evensidemargin=0in
\oddsidemargin=0in
\textwidth=6.5in
\textheight=9.0in
\headsep=0.25in

\linespread{1.1} % Line spacing

% Set up the header and footer
\pagestyle{fancy}
%\lhead{\hmwkAuthorName} % Top left header
\chead{\hmwkClass\ : \hmwkTitle} % Top center head
\rhead{\firstxmark} % Top right header
\lfoot{\lastxmark} % Bottom left footer
\cfoot{} % Bottom center footer
\rfoot{Page\ \thepage\ of\ \protect\pageref{LastPage}} % Bottom right footer
\renewcommand\headrulewidth{0.4pt} % Size of the header rule
\renewcommand\footrulewidth{0.4pt} % Size of the footer rule

\setlength\parindent{0pt} % Removes all indentation from paragraphs

%----------------------------------------------------------------------------------------
%	CODE INCLUSION CONFIGURATION
%----------------------------------------------------------------------------------------

\definecolor{MyDarkGreen}{rgb}{0.0,0.4,0.0} % This is the color used for comments
\lstloadlanguages{Perl} % Load Perl syntax for listings, for a list of other languages supported see: ftp://ftp.tex.ac.uk/tex-archive/macros/latex/contrib/listings/listings.pdf
\lstset{language=Perl, % Use Perl in this example
        frame=single, % Single frame around code
        basicstyle=\small\ttfamily, % Use small true type font
        keywordstyle=[1]\color{Blue}\bf, % Perl functions bold and blue
        keywordstyle=[2]\color{Purple}, % Perl function arguments purple
        keywordstyle=[3]\color{Blue}\underbar, % Custom functions underlined and blue
        identifierstyle=, % Nothing special about identifiers                                         
        commentstyle=\usefont{T1}{pcr}{m}{sl}\color{MyDarkGreen}\small, % Comments small dark green courier font
        stringstyle=\color{Purple}, % Strings are purple
        showstringspaces=false, % Don't put marks in string spaces
        tabsize=5, % 5 spaces per tab
        %
        % Put standard Perl functions not included in the default language here
        morekeywords={rand},
        %
        % Put Perl function parameters here
        morekeywords=[2]{on, off, interp},
        %
        % Put user defined functions here
        morekeywords=[3]{test},
       	%
        morecomment=[l][\color{Blue}]{...}, % Line continuation (...) like blue comment
        numbers=left, % Line numbers on left
        firstnumber=1, % Line numbers start with line 1
        numberstyle=\tiny\color{Blue}, % Line numbers are blue and small
        stepnumber=5 % Line numbers go in steps of 5
}

% Creates a new command to include a perl script, the first parameter is the filename of the script (without .pl), the second parameter is the caption
\newcommand{\perlscript}[2]{
\begin{itemize}
\item[]\lstinputlisting[caption=#2,label=#1]{#1.pl}
\end{itemize}
}

%----------------------------------------------------------------------------------------
%	DOCUMENT STRUCTURE COMMANDS
%	Skip this unless you know what you're doing
%----------------------------------------------------------------------------------------

% Header and footer for when a page split occurs within a problem environment
\newcommand{\enterProblemHeader}[1]{
%\nobreak\extramarks{#1}{#1 continued on next page\ldots}\nobreak
%\nobreak\extramarks{#1 (continued)}{#1 continued on next page\ldots}\nobreak
}

% Header and footer for when a page split occurs between problem environments
\newcommand{\exitProblemHeader}[1]{
%\nobreak\extramarks{#1 (continued)}{#1 continued on next page\ldots}\nobreak
%\nobreak\extramarks{#1}{}\nobreak
}

\setcounter{secnumdepth}{0} % Removes default section numbers
\newcounter{homeworkProblemCounter} % Creates a counter to keep track of the number of problems

\newcommand{\homeworkProblemName}{}
\newenvironment{homeworkProblem}[1][Problem \arabic{homeworkProblemCounter}]{ % Makes a new environment called homeworkProblem which takes 1 argument (custom name) but the default is "Problem #"
\stepcounter{homeworkProblemCounter} % Increase counter for number of problems
\renewcommand{\homeworkProblemName}{#1} % Assign \homeworkProblemName the name of the problem
\section{\homeworkProblemName} % Make a section in the document with the custom problem count
%\enterProblemHeader{\homeworkProblemName} % Header and footer within the environment
}{
%\exitProblemHeader{\homeworkProblemName} % Header and footer after the environment
}

\newcommand{\problemAnswer}[1]{ % Defines the problem answer command with the content as the only argument
\noindent\framebox[\columnwidth][c]{\begin{minipage}{0.98\columnwidth}#1\end{minipage}} % Makes the box around the problem answer and puts the content inside
}

\newcommand{\homeworkSectionName}{}
\newenvironment{homeworkSection}[1]{ % New environment for sections within homework problems, takes 1 argument - the name of the section
\renewcommand{\homeworkSectionName}{#1} % Assign \homeworkSectionName to the name of the section from the environment argument
\subsection{\homeworkSectionName} % Make a subsection with the custom name of the subsection
%\enterProblemHeader{\homeworkProblemName\ [\homeworkSectionName]} % Header and footer within the environment
}{
%\enterProblemHeader{\homeworkProblemName} % Header and footer after the environment
}

%----------------------------------------------------------------------------------------
%	NAME AND CLASS SECTION
%----------------------------------------------------------------------------------------

\newcommand{\hmwkTitle}{Übung\ \#5} % Assignment title
\newcommand{\hmwkDueDate}{Donnerstag,\ 03.\ Dezember\ 2014} % Due date
\newcommand{\hmwkClass}{GPU Computing} % Course/class
\newcommand{\hmwkClassTime}{} % Class/lecture time
\newcommand{\hmwkClassInstructor}{} % Teacher/lecturer
\newcommand{\hmwkAuthorName}{Günther Schindler, Alexander Schnapp, Klaus Naumann} % Your name

%----------------------------------------------------------------------------------------
%	TITLE PAGE
%----------------------------------------------------------------------------------------

\title{
\vspace{2in}
\textmd{\textbf{\hmwkClass:\ \hmwkTitle}}\\
\normalsize\vspace{0.1in}\small{Abgabe\ am\ \hmwkDueDate}\\
\vspace{0.1in}\large{\textit{\hmwkClassTime}}
\vspace{3in}
}

\author{\textbf{\hmwkAuthorName}}
\date{} % Insert date here if you want it to appear below your name

%----------------------------------------------------------------------------------------

\begin{document}

\maketitle

%----------------------------------------------------------------------------------------
%	TABLE OF CONTENTS
%----------------------------------------------------------------------------------------

%\setcounter{tocdepth}{1} % Uncomment this line if you don't want subsections listed in the ToC
\newpage
\tableofcontents
\newpage

%----------------------------------------------------------------------------------------
%	Reading
%----------------------------------------------------------------------------------------

\begin{homeworkProblem}[Reading - The GPU Computing Era]
In "The GPU Computing Era" the authors John Nickolls and William J. Dally illustrate the fast evolution of GPUs from a single-chip accelerator for computer games in the late 90s to a massively parallel GPGPU boosting parallel applications nowadays. In this short period of time the GPU evolved to a programmable high-end computing tool with a scalable parallel architecture including cached memory hierarchy and streaming multiprocessors with several CUDA processor cores each capable of managing thousands of parallel threads through an efficient single-instruction multiple-thread architecture. \\ \\
To achieve the highest application performance a coordinated synergy of CPU and GPU is the best-practice in the parallel application developement. By using coprocessing an application can reach a speedup from 9 up to 137 - depending on the application field. As application becomes more parallel it is important to split up the execution of code between latency-optimized CPUs for serial fractions and throughput-optimized GPUs for parallel fractions to gain the highest efficiency and shortest execution time. \\ \\
Though the article is rather sales promotion for NVIDIA products than a research paper the conclusion shows the possibilities of coprocessing architecture. By using the right processor for the right task an application can benefit from high speedups compared to CPU-only architectures.    
\end{homeworkProblem}
%----------------------------------------------------------------------------------------
%	Matrix Multiply – GPU naïve version
%----------------------------------------------------------------------------------------

\begin{homeworkProblem}[Matrix Multiply – GPU naïve version]
In der fünften Übung ist zunächst eine naive form der Matrix-Multiplikation in CUDA zu 
realisieren. Es soll die Anzahl der Threads-Per-Block (tpb) variiert werden um den optimalen
Parameter dafür zu finden. Es hat sich gezeigt, dass die Multiplikation bei 8 tpb 
die beste Perfomance liefert.

Mit dieser Anzahl an tpb wurde die gesamte Ausführungszeit sowie die einzelnen
Ausführungszeiten für den host-to-device Datentransfer, device-to-host Datentransfer
und die Kernel Ausführung für verschiedene Problemgrößen gemessen. Nachfolgender 
Plot zeigt entsprechende Ergebnisse.
\begin{center}
\includegraphics[width=0.8\columnwidth]{globmatmul.png}
\end{center}
Man sieht, dass sich eine erhebliche Beschleunigung der Ausführungszeit gegenüber der
reinen CPU Version ergibt. Ebenfalls sieht man, dass sich die gesamte Auführungszeit 
für große Problemgrößen an die Kernel Ausführungszeit annähert. Dies bedeutet, dass
der zusätzliche Datentransfer für große Problemgrößen kaum eine Rolle spielt.

Abhängig von der Problemgröße und davon ob die zusätzliche Zeit für den Datentransfer
mit einberechnet wird, ergibt sich folgender Verlauf für den Speed-Up (im Vergleich 
zur CPU-Version).
\begin{center}
\includegraphics[width=0.8\columnwidth]{speed-up_glob.png}
\end{center}
Auch hier wird deutlich, dass die zusätzliche Zeit für den Datentransfer nur für
kleine Problemgrößen relevant ist. Für große Problemgrößen konnte hier ein
Speed-Up von bis zu 33 erzielt werden.
\end{homeworkProblem}
%----------------------------------------------------------------------------------------
%	Matrix Multiply – GPU version using shared memory
%----------------------------------------------------------------------------------------
\begin{homeworkProblem}[Matrix Multiply – GPU version using shared memory]
Als nächstes ist die naive CUDA Version so zu erweitern, dass diese den Shared
Memory eines jeden Streaming Multiprocessors ausnutzt. Es ist wieder der
optimale Wert für die tpb zu ermitteln. Nachfolgender Plot zeigt die
Ausführungszeit für eine 2048x2048 Matrix in Abhängigkeit von den verwendeten
tpb.
\begin{center}
\includegraphics[width=0.8\columnwidth]{var_tpb.png}
\end{center}
Die Grafik zeigt, dass mit steigender Anzahl von tpb auch die Ausführungszeit
sinkt. Somit ist der optimale Wert für die tpb 32.

Mit dieser Anzahl an tpb wurde wieder die gesamte Ausführungszeit sowie die einzelnen
Ausführungszeiten für den host-to-device Datentransfer, device-to-host Datentransfer
und die Kernel Ausführung für verschiedene Problemgrößen gemessen. Nachfolgender 
Plot zeigt entsprechende Ergebnisse.
\begin{center}
\includegraphics[width=0.8\columnwidth]{shmatmul.png}
\end{center}
Die abschließende Grafik zeigt noch den erreichten Speed-Up verglichen mit der CPU 
und der Global Memory Version und mit bzw. ohne Datentransfer, abhängig von der
Problemgröße.
\begin{center}
\includegraphics[width=0.8\columnwidth]{speed-up_sh.png}
\end{center}
Obwohl sich die Ausführungszeiten der Matrix-Multiplikation zwischen Global und 
Shared Memory wenig unterscheiden, konnte hier ein Speed-Up von knapp 3 
erreicht werden. Bezüglich der CPU Version konnte ein maximaler Speed-Up von
bis zu 70 erreicht werden.
\end{homeworkProblem}
\pagebreak
\end{document}